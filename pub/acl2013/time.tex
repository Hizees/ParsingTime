\Section{time}{Temporal Representation}
% -- Overview
We define a compositional representation of time, similar to \me, but with
  a greater focus on efficiency and simplicity.
The representation makes use of a notion of temporal \textit{types} and
  their associated semantic \textit{values};
  a grammar is constructed over these types, and is grounded by appealing
  to the associated values.

% -- Outline
A summary of the temporal type system is provided in \refsec{types-review};
  the grammar is described in \refsec{grammar-review};
  key modifications from previous work are highlighted in \refsec{differences}.

% -----
% TYPES
% -----
\Subsection{types-review}{Temporal Types}
% -- Intro to Types
Temporal expressions are represented either as a \ty{Range}, \ty{Sequence},
	or \ty{Duration}.
In addition, phrases can be tagged as a \ty{Function}; or, as a
  special \ty{Nil} type corresponding to segments 
  without a direct temporal interpretation.
We describe each of these briefly below.

% -- Range
\paragraph{Range [and Instant]}
%(definition)
A period between two dates (or times), as per an interval-based theory of
  time \cite{key:1981allen-temporal}.
%(examples)
This includes entities such as \te{Today}, \te{1987}, or
	\te{Now}.

% -- Sequence Grounding
\FigStar{fig/sequence.pdf}{0.35}{sequence-grounding}{
  An illustration of grounding a \ty{Sequence}.
  When grounding the \ty{Sequence} \te{November \th{27}} with a reference time
    \te{2013-08-06 03:25:00}, we complete the missing fields in the
    \ty{Sequence} (the year) with the corresponding field in the reference
    time (2013).
}

% -- Sequence 
\paragraph{Sequence}
%(definition)
A sequence of \ty{Range}s, occuring at regular but not necessarily constant
  intervals.
%(examples)
This includes entities such as \te{Friday}, \te{November \th{27}}, or
	\te{last Friday}.
%(anatomy)
A \ty{Sequence} is defined in terms of a partial completion of calendar fields.
For example, \te{November \th{27}} would define a \ty{Sequence} whose
  year is unspecified,
  month is November, and day is the \th{27}; spanning the entire range of the
  lower order fields (hour, minute, second).
This example is illustrated in \reffig{sequence-grounding}.
Note that a \ty{Sequence} implicitly selects an infinite number of possible 
  grounded \ty{Range}s.

%(definition of reference time)
To select a particular grounded time for a \ty{Sequence}, 
  we appeal to a notion of a
  \textit{reference time} \cite{key:1947reichenback-temporal}.
For the \tempeval\ corpus, we approximate this as the
	publication time of the article.
While this is conflating Reichenbach's reference time with speech time, 
  and comes at the expense of certain mistakes,
	it is a nonetheless useful in practice.

%(grounding)
To a first approximation, grounding a sequence given a reference time
  corresponds to filling in the unspecified fields of the sequence with the
  fully-specified fields of the reference time.
Due to the irregular nature of time, this process has a number of special
  cases not enumerated here,\footnote{
    Some of these special cases are caused by variable days of the month,
      daylight savings time, etc.
    Another class arises from pragmatically peculiar utterances; e.g.,
      \tp{the next Monday in August} uttered in the last week of August
      should ground to August of next year (rather than the reference time's
      year).
  }
  but the process remains feasible in constant time.
	
% -- Duration
\paragraph{Duration}
%(definition)
A period of time.
%(examples)
This includes entities like \te{Week}, \te{Month}, and \te{7 days}.
%(fuzzy)
A special case of the \ty{Duration} type is defined to represent 
	\textit{approximate} durations.
These are used to represent expressions such as \tp{a few years} or
	\tp{some days}.

% -- Functions
\paragraph{Function}
%(definition)
A function of arity less than or equal to two representing some
	general modification to one of the above types.
%(examples)
This captures semantic entities such as those implied in
	\tp{last $x$}, 
	\tp{the third $x$ [of $y$]}, or \tp{$x$ days ago}.
%(etc.)
The particular functions and their application are enumerated in 
	\reftab{function}.

% -- Nil
\paragraph{Nil}
Lastly, a special \ty{Nil} type denotes terms which are not
	directly contributing to the semantic meaning of the expression.
This is intended for words such as \tp{a} or \tp{the}, which serve as cues
	without bearing temporal content themselves.


% -----
% GRAMMAR
% -----
\Subsection{grammar-review}{Temporal Grammar}
% -- Time is compositional
% (compositional)
Our approach assumes that natural language descriptions
	of time are compositional in nature; that is,
  each word attached to a temporal phrase is 
	compositionally modifying the meaning of the phrase.
% (pcfg)
We define a grammar jointly over temporal \textit{types} and
  \textit{values}.
The types serve to constrain the parse and allow for coarse features over the
  parse;
  the values encode the specific semantics of the parse, and allow for finer
  features.
At the root of a parse tree, we recursively apply the functions in the tree
	to obtain a final temporal value.

% (compare)
This approach can been presented as a rule-to-rule translation
	\cite[p.~263]{key:1976bach-semantics,key:1995allen-semantics},
	or a constrained Synchronous PCFG
	\cite{key:2001yamada-syntaxmt}.
However, in the discriminative setting we can also cast it as an infinite
  grammar over temporal values with type-checking enforced.


% -- Formal definition
Formally, we define our  grammar as
	\mbox{$G=\left(\Sigma,S,\sV,\sW,\sR\right)$}.
The alphabet $\Sigma$ and start symbol $S$ retain their usual interpretations.
% (nonterminals)
We define a set $\sV$ to be the set of types, as described in
	\refsec{types-review} -- these act as our nonterminals.
For each $v \in \sV$ we define an (infinite) set $W_v$ corresponding to the 
	possible instances of type $v$.
Each node in the tree defines a pair $(v,w)$ such that $w \in W_v$,
	with combination rules defined over $v$ and function applications performed
	on $w$.
% (rules)
A rule \mbox{$R \in \sR$} is defined as a pair 
	\mbox{$R = \left(v_i \rightarrow v_jv_k, 
		f : (W_{v_j},W_{v_k}) \rightarrow W_{v_i}\right)$}.
Note that this definition is trivially adapted for the case of unary rules.

% -- Static Lexicon
\begin{table}[tb]
	\begin{center}
	\begin{tabular}{|l|l|}
		\hline
		\textbf{Type} & \textbf{Example Instances} \\
		\hline
		\hline
		% -- Ranges
		\ty{Range} &
			\te{Past}, \te{Future}, \te{Yesterday}, \\
			& \te{Tomorrow}, \te{Today}, \te{Reference}, \\
			& \te{Year($n$)}, \te{Century($n$)} \\
		\hline
		% -- Sequences
		\ty{Sequence} 
			%(named sequences)
			& \te{Friday}, \te{January}, $\dots$ \\
			& \te{DayOfMonth}, \te{DayOfWeek}, $\dots$ \\
			& \te{EveryDay}, \te{EveryWeek}, $\dots$ \\
		\hline
		% -- Durations
		\ty{Duration}
			& \te{Second}, \te{Minute}, \te{Hour}, \\
			& \te{Day}, \te{Week}, \te{Month}, \te{Quarter}, \\
			& \te{Year}, \te{Decade}, \te{Century} \\
		\hline
	\end{tabular}
	\caption{
		The content-bearing preterminals of the grammar, arranged by their
			types.
		Note that the \ty{Sequence} type contains more elements
			than enumerated here; however, only a few of each characteristic type
			are shown here for brevity.
	}
	\label{tab:content}
	\end{center}
\end{table}

% -- Function Lexicon
\begin{table}[t]
	\begin{center}
	\begin{tabular}{|l|l|}
		\hline
		\textbf{Function} & \textbf{Description} \\
		\hline
		\hline
		% -- Arity 2
		%(shift left)
		{shiftLeft}
			& Shift a \ty{Range} left by a \ty{Duration} \\
		{shiftRight}
			& Shift a \ty{Range} right by a \ty{Duration} \\
		\hline
		%(shrink begin)
		{shrinkBegin} 
			& Take the first \ty{Duration} of a \ty{Range} \\
		{shrinkEnd}
			& Take the last \ty{Duration} of a \ty{Range} \\
		\hline
		{catLeft}
			& Take the \ty{Duration} after a \ty{Range} \\
		{catRight}
			& Take the \ty{Duration} before a \ty{Range} \\
		\hline
		% -- Arity 1
		%(seq move)
		{moveLeft1}
			& Shift a \ty{Sequence} left by 1 \\
		{moveRight1}
			& Shift a \ty{Sequence} right by 1 \\
		\hline
		%(n^th element)
		{\th{$n$} $x$ of $y$}
			& Take the \th{$n$} element in $y$ \\
		\hline
		%(fuzzify)
		{approximate}
			& Make a \ty{Duration} approximate \\
		\hline
	\end{tabular}
	\caption{
		The functional preterminals of the grammar.
		The name and a brief description of the function are given;
		  the functions are most
			easily interpreted as operations on either an interval or sequence.
    All operations on \ty{Range}s can equivalently be applied
      to \ty{Sequence}s.
	}
	\label{tab:function}
	\end{center}
\end{table}

% -- Preterminals
We keep the preterminals identical to those in \me.
Each preterminal consists of a \textit{type} and a \textit{value};
  neither which are lexically informed.
That is, the word \tp{week} and preterminal $(\te{Week}, \ty{Duration})$
  are not tied in any way.
A total of 62 preterminals are defined corresponding to instances of
  \ty{Range}s, \ty{Sequence}s, and \ty{Duration}s; these are summarized
  in \reftab{content}.

In addition, 10 functions are defined for manipulating temporal expressions.
These are summarized in \reftab{function}.
The majority of these mirror generic operations on intervals on a timeline,
	or manipulations of a sequence.
Notably, like intervals, times can be moved (\tp{3 weeks ago}) or
	their size changed (\tp{the first two days of the month}), or
	a new interval can be started from one of the endpoints
	(\tp{the last 2 days}).
Additionally, a sequence can be modified by shifting its origin
	(\tp{last Friday}), 
	or taking the \th{$n$} element of the sequence within some bound
	(\tp{fourth Sunday in November}).

% -- Combination
Combination rules in the grammar derive from type-checked
  curried function application.
For instance, the function \te{moveLeft1} applied
  to \te{week} (as in \tp{last week}) yields a grammar rule:
\begin{align*}
&  (\ty{Sequence}, \te{moveLeft1$($week$)$}) \rightarrow \\
&~~~(f:\ty{Sequence$\rightarrow$Sequence}, \te{moveLeft1}) \\
&~~~(\ty{Sequence}, \te{week})
\end{align*}
In more generality, we create grammar rules for applying a function
  on either the left or the right: $f(x,y) \odot x$ or $x \odot f(x,y)$.

% -- Combination
Additionally, a grammar rule is created for intersecting two \ty{Range}s or
  \ty{Sequence}s; for multiplying a duration by a number; and for absorbing
  a \ty{Nil} span.
The first two rules can be though of as an implicit function application.
Absorbing \ty{Nil} spans is in turn the application of the identity function.

% -----
% DIFFERENCES
% -----
\Subsection{differences}{Differences From Previous Work}
While the grammar formalism is strongly inspired by \me, a number of changes
  have been made to both simplify the framework, and make inference more
  efficient.

\paragraph{Sequence Grounding}
% -- Time consuming
% (overview)
The most time-consuming and conceptually nuanced aspect of temporal inference
  in \me\ was intersecting \ty{Sequence}s.
In particular, our calendar has two modes of expressing
  dates which resist intersection: a day-of-month-based mode and a
  week-based mode.
Properly grounding a sequence which defines both a day of the month and
  a day of the week (or week of the year) requires backing
  off to an expensive search problem.

% (example)
To illustrate, consider the example: \tp{Friday the \th{13}}.
Although both a Friday and a \th{13} of the month are easily found, the
  intersection of the two requires iterating through elements of one until
  it overlaps with an element of the other.
This requires time proportional to the log of the distance to the next match,
  measured in the smallest interval between sequences being intersected.

% (efficiency hit)
When considering that elements in the beam of possible groundings can become
  both complex and pragmatically unreasonable this can result in a noticable
  efficiency hit; e.g., a sentence could have a
  [likey incorrect] candidate interpretation of:
  \tp{nineteen ninety-six Monday the \th{5} from now}.
In this implementation such intersections are disallowed, in the same fashion
  as February \th{30} would be.

\paragraph{Sequence Pragmatics}
% -- Pragmatic Distribution
For the sake of simplicity the pragmatic distribution over
  possible groundings of a sequence is replaced with the single most likely
  grounding, as trained by \me\ on the English \tempeval\ corpus.

% -- Numbers
\paragraph{Numbers}
The \ty{Number} type no longer appears explicitly, but is rather folded
  into the features of the parse (see \refsec{features}).

% -- Parameters
\paragraph{Grammar Parameters}
As this model is discriminative rather than generative, the grammar
  parameters $\theta$ are defined over features rather than over the
  probabilities of rules.


% -- Segway
We proceed to describe learning in our system.
