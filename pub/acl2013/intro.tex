\Section{intro}{Introduction}
% -- Intro
Temporal resolution is the task of mapping from a textual phrase describing
	a potentially complex time, date, or duration to a normalized
	(\textit{grounded}) temporal representation.
For example, possibly complex phrases such as \tp{the week before last} are
	often more useful in their grounded form -- e.g., \te{January 1 - January 7}.

% -- Why Parsing
%(intro to systems)
Many approaches to this problem make use of rule-based methods, combining
  regular-expression matching and hand-written interpretation functions.
%(motivation)
In general, it is appealing to learn the interpretation of temporal expressions,
	rather than hand-building systems.
Probabilistic systems furthermore allow propagation of uncertainty
	to higher-level components.
Importantly, a probabilistic system could back-off to a rule-based alternative
  for uncertain parses, and visa versa.

% -- Why Language Independent
A large number of languages conceptualize time as lying on a one dimensional
  line.
Although the surface form of temporal expressions differ, the basic operations
  the language expresses are operations on this time line (see \refsec{time}).
Furthermore, many common languages share common units (hours, weekdays, etc.).
By structuring a latent parse to reflect these semantics, we can define a single
  model which performs well on multiple languages.

% -- Training Data
Our system requires annotated data consisting only of an input phrase and
  an associated \textit{grounded} time, relative to some reference time.
Training data of this distantly-supervised form
  is generally easier to collect than the alternative of
  manually creating and tuning potentially complex interpretation rules.

% -- Why Discriminative
A discriminative parsing model allows us to capture sparse features over
  the semantics of the parse, as well as features capturing richer lexical cues.
For example, it allows us to learn that we are much more likely to
  express \tp{March \th{14}} than \tp{2pm in March} -- despite the fact that
  both interpretations are composed of similar types of components.
Furthermore, it allows us to define both sparse n-gram and denser but less
  informative bag-of-words features over multi-word phrases, and allows us
  to handle numbers in a more flexible way.

%(table of contents)
We briefly review our temporal representation,
	followed by a description of the learning algorithm; 
	we conclude with experimental results on the languages of the
  \tempeval\ A task.



