
\Section{intro}{Introduction}
% -- Intro
Temporal resolution is the task of mapping from a textual phrase describing
	a potentially complex time, date or duration expression, to a normalized
	(\textit{grounded}) temporal representation.
For example, possibly complex pharses such as \tp{the week before last} are
	often more useful in their grounded form -- e.g., \te{January 1 - January 7}.

%% -- Table
%\begin{center}
%\begin{tabular}{ccccc}
%\textit{(Reference Time)} & \textbf{Phrase} & & \textbf{Grounding} \\
%{\color{blue}11/24/2011} 
%	& \tp{\color{darkgreen}Sunday} 
%		& $\rightarrow$ 
%	& \color{darkred}11/27/2011 \\
%\end{tabular}
%\end{center}

% -- Temporal Comparison
%(intro to systems)
The dominant approach to this problem in previous work has been to use
	rule-based methods, generally a combination of regular-expression matching
	followed by hand-written interpretation functions
	% Edinburgh, TRIPS/TRIOS, TERSEO, KUL, Marta's Friend, random, Heideltime
	\cite{key:2000mani-temporal,key:2003saquete-temporal,key:2004puscasu-temporal,key:2010grover-temporal,key:2010uzzaman-temporal,key:2010kolomiyets-temporal,key:2010strotgen-temporal}.

% -- Motivation
%(motivation)
In general, it is appealing to learn the interpretation of temporal expressions,
	rather than hand-building systems.
Moreover,
	we note that complex temporal expressions such as \tp{the Tuesday before last}
	or \tp{the third Wednesday of each month} are poorly handled by current
	systems and suggest the need for recursive phrase structure representations.
%(approach)
Therefore, in contrast to previous rule-based approaches, we attempt to learn
	a temporal interpretation system where temporal phrases are parsed by
	a grammar, but this grammar and its semantic interpretation rules are
	latent, with only the input phrase and its grounded interpretation given
	to the learning system.

% -- Approach
%(ambiguity)
Employing probabilistic techniques allows us to capture ambiguity in temporal 
	phrases in two important respects.
In part, it captures syntactic ambiguity -- as in \tp{last Friday
	the \th{13}} bracketing as either \tp{[last Friday] [the \th{13}]}, or
	\tp{last [Friday the \th{13}]}.
This also includes examples of lexical ambiguity -- e.g., two
	meanings of \tp{last} in \tp{last week of November} verus \tp{last week}.
In addition, temporal expressions often carry a pragmatic ambiguity.
For instance, a speaker may refer to either the next or previous Friday
	when he utters \tp{Friday} on a Sunday.
Furthermore, probabilistic systems in general allow propagation of uncertainty
	to higher-level components -- for example recognizing that \tp{May} could
	have a number of non-temporal meanings and allowing a system with a broader
	contextual scope to make the final judgment.

%(table of contents)
We describe our temporal representation,
	followed by the learning algorithm; 
	we conclude with experimental results showing our
	approach to be competitive with state of the art systems.


