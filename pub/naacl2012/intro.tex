
\Section{intro}{Introduction}
% -- Intro
Temporal resolution is the task of mapping from a textual phrase describing
	a potentially complex time, date or duration expression, to a normalized
	(\textit{grounded}) temporal representation.
For example, possibly complex phrases such as \tp{the week before last} are
	often more useful in their grounded form -- e.g., \te{January 1 - January 7}.

%% -- Table
%\begin{center}
%\begin{tabular}{ccccc}
%\textit{(Reference Time)} & \textbf{Phrase} & & \textbf{Grounding} \\
%{\color{blue}11/24/2011} 
%	& \tp{\color{darkgreen}Sunday} 
%		& $\rightarrow$ 
%	& \color{darkred}11/27/2011 \\
%\end{tabular}
%\end{center}

% -- Temporal Comparison
%(intro to systems)
The dominant approach to this problem in previous work has been to use
	rule-based methods, generally a combination of regular-expression matching
	followed by hand-written interpretation functions.

% -- Motivation
%(motivation)
In general, it is appealing to learn the interpretation of temporal expressions,
	rather than hand-building systems.
Moreover, complex hierarchical temporal expressions such as 
	\tp{the Tuesday before last} or \tp{the third Wednesday of each month},
	and ambiguous expressions such as \tp{last Friday} are difficult to
	handle using deterministic rules and would benifit from a
	recursive and probabilistic phrase structure representations.
%(approach)
Therefore, we attempt to learn
	a temporal interpretation system where temporal phrases are parsed by
	a grammar, but this grammar and its semantic interpretation rules are
	latent, with only the input phrase and its grounded interpretation given
	to the learning system.

% -- Approach
%(ambiguity)
Employing probabilistic techniques allows us to capture ambiguity in temporal 
	phrases in two important respects.
In part, it captures syntactic ambiguity -- as in \tp{last Friday
	the \th{13}} bracketing as either \tp{[last Friday] [the \th{13}]}, or
	\tp{last [Friday the \th{13}]}.
This also includes examples of lexical ambiguity -- e.g., two
	meanings of \tp{last} in \tp{last week of November} versus \tp{last week}.
In addition, temporal expressions often carry a pragmatic ambiguity.
For instance, a speaker may refer to either the next or previous Friday
	when he utters \tp{Friday} on a Sunday.
Similarly, \tp{next week} can refer to either the coming week or the week
	thereafter.

Probabilistic systems furthermore allow propagation of uncertainty
	to higher-level components -- for example recognizing that \tp{May} could
	have a number of non-temporal meanings and allowing a system with a broader
	contextual scope to make the final judgment.
In order to demonstrate our model's ability to incorporate with such components,
	and to allow for straightforward comparisons with the other systems in the
	\tempeval\ task, we incorporate our model into a Conditional Random Field
	\cite{key:2001lafferty-crf} to allow for detection as well as interpretation
	of temporal expressions.

%(table of contents)
We describe our temporal representation,
	followed by the learning algorithm; 
	we conclude with experimental results showing our
	approach to be competitive with state of the art systems.


