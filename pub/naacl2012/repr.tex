\Section{repr}{Representation}
We define a compositional representation of time; a type system 
	is described in \refsec{repr-types} while the grammar is outlined in
	\refsec{repr-compositional} and described in detail in 
	\refsecs{repr-lexicon}{repr-rules}

%%%%%%%%%%%%%%%%%%%%%%%%%%%%%%%%%%%%%%%%%%%%%%%%%%%%%%%%%%%%%%%%%%%%%%%%%%%%%%%%
% TEMPORAL EXPRESSIONS
%%%%%%%%%%%%%%%%%%%%%%%%%%%%%%%%%%%%%%%%%%%%%%%%%%%%%%%%%%%%%%%%%%%%%%%%%%%%%%%%
\Subsection{repr-types}{Temporal Expression Types}
% -- Intro to Types
We represent temporal expressions as either a \ty{Range}, \ty{Sequence},
	or \ty{Duration}.
We describe these, the \ty{Function} type, and the miscellaneous
	\ty{Number} and \ty{Nil} types below:
% -- Range
\paragraph{Range [and Instant]}
%(definition)
A period between two dates (or times).
%(examples)
This includes entities such as \te{Today}, \te{1987}, or
	\te{Now}.
We denote a range by the variable \range.
We maintain a consistent interval-based theory of time
	\cite{key:1981allen-temporal} and represent instants as intervals with
	zero span.

% -- Sequence 
\paragraph{Sequence}
%(definition)
A sequence of \ty{Range}s, not necessarily occurring at regular intervals.
%(examples)
This includes entities such as \te{Friday}, \te{November \th{27}}, or
	\te{last Friday}.
%(anatomy)
A \ty{Sequence} is a tuple of three elements
	$\seq=(\seqelem,\seqjump,\seqcanon)$:
\begin{enumerate}
	\setlength{\itemsep}{-5pt} 
	\item $\seqelem(i)$: 
		The \th{i} element of a sequence, of type \ty{Range}.
		In the case of the sequence \te{Friday}, $\seqelem(0)$ corresponds to
			\tp{the Friday in the current week}; 
			$\seqelem(1)$ is the Friday in the following week, etc.
	\item \seqjump:
		The \textit{distance} between two elements in the sequence -- approximated
			if this distance is not constant.
		In the case of \te{Friday}, this distance would be a \te{Week}.
	\item \seqcanon:
		The \textit{containing unit} of an element of a sequence.
		For example, \seqcanonX{Friday} would be the \ty{Range}
			corresponding to \tp{the current week}.
		The sequence index $i \in \BZ$ is defined relative to $\seqelem(0)$ --
			the element in the same containing unit as the reference time.
\end{enumerate}

%(definition of reference time)
We define the \textit{reference time} $t$ \cite{key:1947reichenback-temporal}
	to be the instant which is considered to be the current time relative to
	to the expression uttered.
For the \tempeval\ corpus, we approximate this (occasionally incorrectly) as the
	publication time of the article.

%(sequence as distribution)
To contrast with \ty{Range}s, a \ty{Sequence} can represent a number of grounded
	times.
Nonetheless, pragmatically, not all of these are given equal weight --
	an utterance of \tp{last Friday} may mean either of the previous two Fridays,
	but is unlikely to ground to anything else.
We represent this ambiguity by defining a distribution over the elements
	of the \ty{Sequence}.
While this could be any distribution, we chose to approximate it as a
	Gaussian.
The domain of the Gaussian is the distance between the reference time
	and the element in of the sequence in question; to account for sequences
	of different granularities (e.g., \te{Friday} versus \te{\th{13}}),
	we normalize this distance by \seqjump.
Our probability, with learned parameters $\mu$ and $\sigma$, thus becomes:
% -- Gaussian Plot
\begin{figure}[t]
\begin{center}
	\resizebox{1.1\hsize}{!}{
		\input fig/gauss.tex
	}
	\caption{
		\label{fig:distribution}
		An illustration of a temporal distribution, e.g., \te{Sunday}.
		The reference time is labeled as time $t$ between \te{Nov 20} and 
			\te{Nov 27}; the probability that this sequence is referring to
			\te{Nov 20} is integral of the marked area.
		Note that shifting the reference time towards \te{Nov 20} would increase
			this probability.
	}
\end{center}
\end{figure}
% -- Gaussian Equation
\begin{equation}
\label{eqn:gaussian}
	P_t(i) = 
	\int_{i-0.5}^{i+0.5}
		\sN_{\mu,\sigma}\left(\frac{\delta_{s,t}(i)-\delta_{s,t}(0)}{\seqjump}\right)
\end{equation}

%(global scope)
\reffig{distribution} shows an example of such a distribution;
	importantly, note that moving the reference time between two elements
	dynamically changes the probability assigned to each.
	
% -- Duration
\paragraph{Duration}
%(definition)
A period of time.
%(examples)
This includes entities like \te{Week}, \te{Month}, and \te{7 days}.
%(etc.)
We denote a duration with the variable \dur.

%(fuzzy)
We define a special case of the \ty{Duration} type to represent 
	\textit{approximate} durations, identified by their cannonical
	unit (\te{week}, \te{month}, etc).
These are used to represent expressions such as \tp{a few years} or
	\tp{some days}.

% -- Functions
\paragraph{Function}
%(definition)
A function of arity less than or equal to two representing some
	general modification to one of the above types.
%(examples)
This captures semantic entities such as those implied in
	\tp{last $x$}, 
	\tp{the third $x$ [of $y$]}, or \tp{$x$ days ago}.
%(etc.)
The particular functions and their application are enumerated in 
	\reftab{function}.


% -- Other
\paragraph{Other Types}
Two other types bear auxiliary roles in representing temporal expressions,
	though they are not directly temporal concepts.
In the grammar, these appear as preterminals only.

The first of these types is \ty{Number} -- denoting a number without
	any temporal meaning attached.
This comes into play representing expressions such as \tp{2 weeks}.
The other is the \ty{Nil} type -- denoting terms which are not
	directly contributing to the semantic meaning of the expression.
This is intended for words such as \tp{a} or \tp{the}, which serve as cues
	without bearing temporal content themselves.
The \ty{Nil} type is lexicalized with the word it generates.


\paragraph{Omitted Phenomena}
The representation described is a simplification of the complexities of
	time.
Notably, a body of work has focused on reasoning about events or states
	relative to temporal expressions.
\newcite{key:1988moens-temporal} describes temporal expressions relating to
	changes of state; 
	\newcite{key:2010condoravdi-temporal} explores NPI licensing in temporal
	expressions.
Broader context is also not directly modeled, but rather left to systems
	in which the model would be embedded.
Furthermore, vague times (e.g., \tp{in the 90's}) represent a notable chunk of 
	temporal expressions uttered.
In contrast, NLP evaluations have generally not handled such vague 
	time expressions.

%%%%%%%%%%%%%%%%%%%%%%%%%%%%%%%%%%%%%%%%%%%%%%%%%%%%%%%%%%%%%%%%%%%%%%%%%%%%%%%%
% TIME AS COMPOSITIONAL PARSING
%%%%%%%%%%%%%%%%%%%%%%%%%%%%%%%%%%%%%%%%%%%%%%%%%%%%%%%%%%%%%%%%%%%%%%%%%%%%%%%%
\Subsection{repr-compositional}{Grammar Formalism}
% -- Time is compositional
Our approach builds on the assumption that natural language descriptions
	of time are compositional in nature.
Each word attached to a temporal phrase is 
	usually compositionally modifying the meaning of the phrase.
To demonstrate, we consider the expression \tp{the week before last week}.
We can construct a meaning by 
	applying the modifier \tp{last} to \tp{week} -- 
	creating the previous week;
	and then applying \tp{before} to \tp{week} and \tp{last week}.

% -- Grammar
\Fig{fig/grammar}{0.45}{grammar}{
	The grammar -- (a) describes the CFG parse of the
		temporal \textit{types}. Words are tagged with their nonterminal
		entry, above which only the types of the expressions are maintained;
	(b) describes the corresponding combination
			of the temporal \textit{instances}.
	The parse in (b) is deterministic given the grammar combination rules
		in (a).
}
%(introduction)
We construct a paradigm for parsing temporal phrases 
	consisting of a standard PCFG over temporal \textit{types} with each parse 
	rule defining a function to apply to the child nodes, or the word being
	generated.
At the root of the tree, we recursively apply the functions in the parse tree
	to obtain a final temporal \textit{value}.
One can view this formalism as a constrained Synchronous PCFG
	\cite{key:2001yamada-syntaxmt}, or a rule-to-rule translation
	\cite{key:1976bach-semantics}.

Our approach contrasts with common approaches, such as CCG grammars 
	\cite{key:2000steedman-ccg,key:2004bos-ccg,2011kwiatkowski-semantics},
	giving us more flexibility in the composition rules.
\reffig{grammar} shows an example of the grammar.

%(formal definition)
Formally, we define a grammar
	$G = \left(\Sigma,S,\sV,\sW,\sR,\theta\right)$.
The alphabet $\Sigma$ and start symbol $S$ retain their usual interpretations.
%(nonterminals)
We define a set $\sV$ to be the set of types, as described in
	\refsec{repr-types} -- these act as our nonterminals.
For each $v \in \sV$ we define an (infinite) set $W_v$ corresponding to the 
	possible instances of type $v$.
Concretely, if $v=\ty{Sequence}$, our set $W_v \in \sW$ could contain elements
	corresponding to \tp{Friday}, \tp{last Friday}, \tp{Nov. \th{27}}, etc.
Each node in the tree defines a pair $(v,w)$ such that $w \in W_v$,
	with combination rules defined over $v$ and function applications performed
	on $w$.

%(rules)
A rule $R \in \sR$ is defined as a pair 
	$R = \left(v_i \rightarrow v_jv_k, 
		f : (W_{v_j},W_{v_k}) \rightarrow W_{v_i}\right)$.
The first term is our conventional PCFG rule over the types $\sV$.
The second term defines the function to apply to the values returned
	recursively by the child nodes.
Note that this definition is trivially adapted for the case of unary rules.

%(probabilities)
The last term in our grammar formalism denotes the rule probabilities $\theta$.
In line with the usual interpretation, this defines a probability
	of applying a particular rule $r \in R$.
Importantly, note that the distribution over possible groundings
	of a temporal expression are not included in the grammar
	formalism.
The learning of these probabilities is detailed in \refsec{learn}.


%%%%%%%%%%%%%%%%%%%%%%%%%%%%%%%%%%%%%%%%%%%%%%%%%%%%%%%%%%%%%%%%%%%%%%%%%%%%%%%%
% LEXICON
%%%%%%%%%%%%%%%%%%%%%%%%%%%%%%%%%%%%%%%%%%%%%%%%%%%%%%%%%%%%%%%%%%%%%%%%%%%%%%%%
\Subsection{repr-lexicon}{Preterminals}
% -- Static Lexicon
\begin{table}[tb]
	\begin{center}
	\begin{tabular}{|l|l|}
		\hline
		\textbf{Type} & \textbf{Instances} \\
		\hline
		\hline
		% -- Ranges
		\ty{Range} &
			\te{Past}, \te{Future}, \te{Yesterday}, \\
			& \te{Tomorrow}, \te{Today}, \te{Reference}, \\
			& \te{Year($n$)}, \te{Century($n$)} \\
		\hline
		% -- Sequences
		\ty{Sequence} 
			%(named sequences)
			& \te{Friday}, \te{January}, $\dots$ \\
			& \te{DayOfMonth}, \te{DayOfWeek}, $\dots$ \\
			& \te{EveryDay}, \te{EveryWeek}, $\dots$ \\
%			& \te{Morning}, \te{Afternoon}, $\dots$ \\
%			& \te{Monday}, \te{Tuesday}, $\dots$ \\
%			& \te{January}, \te{February}, $\dots$ \\
%			& \te{Winter}, \te{Spring}, $\dots$ \\
%			%(from number)
%			& \te{MinuteOfHour($0 \dots 59$)}, \\
%			& \te{HourOfDay($0 \dots 23$)}, \\
%			& \te{DayOfMonth($1 \dots 31$)}, \\
%			& \te{MonthOfYear($1 \dots 12$)}, \\
%			& \te{YearOfCentury($0 \dots 99$)}, \\
%			& \te{DecadeOfCentury($0 \dots 9$)}, \\
%			& \te{YearOfDecade($0 \dots 9$)}, \\
%			%(durational)
%			& \te{EveryDay}, \te{EveryWeek}, \\
%			& \te{EveryMonth}, \te{EveryQuarter}, \\
%			& \te{EverySeason} \\
		\hline
		% -- Durations
		\ty{Duration}
			& \te{Second}, \te{Minute}, \te{Hour}, \\
			& \te{Day}, \te{Week}, \te{Month}, \te{Quarter}, \\
			& \te{Year}, \te{Decade}, \te{Century} \\
		\hline
	\end{tabular}
	\caption{
		The content-bearing preterminals of the grammar, arranged by their
			types.
		Note that the \ty{Sequence} type contains more elements
			than enumerated here; however, only a few of each characteristic type
			are shown here for brevity.
	}
	\label{tab:content}
	\end{center}
\end{table}

% -- Function Lexicon
\begin{table*}[t]
	\begin{center}
	\begin{tabular}{|l|l|l|}
		\hline
		\textbf{Function} & \textbf{Description} & \textbf{Signature(s)} \\
		\hline
		\hline
		% -- Arity 2
		%(shift left)
		\texttt{shiftLeft}
			& Shift a \ty{Range} or \ty{Sequence} left by a \ty{Duration}
			& $f : \ty{S},\ty{D} \rightarrow \ty{S}$; ~~
			  $f : \ty{R},\ty{D} \rightarrow \ty{R}$ \\
		\texttt{shiftRight}
			& Shift a \ty{Range} or \ty{Sequence} right by a \ty{Duration}
			& $f : \ty{S},\ty{D} \rightarrow \ty{S}$; ~~
			  $f : \ty{R},\ty{D} \rightarrow \ty{R}$ \\
		\hline
		%(shrink begin)
		\texttt{shrinkBegin}
			& Take the first \ty{Duration} of a \ty{Range}/\ty{Sequence}
			& $f : \ty{S},\ty{D} \rightarrow \ty{S}$; ~~
			  $f : \ty{R},\ty{D} \rightarrow \ty{R}$ \\
		\texttt{shrinkEnd}
			& Take the last \ty{Duration} of a \ty{Range}/\ty{Sequence}
			& $f : \ty{S},\ty{D} \rightarrow \ty{S}$; ~~
			  $f : \ty{R},\ty{D} \rightarrow \ty{R}$ \\
		\hline
		\texttt{catLeft}
			& Take \ty{Duration} units after the end of a \ty{Range}
			& $f : \ty{R},\ty{D} \rightarrow \ty{R}$ \\
		\texttt{catRight}
			& Take \ty{Duration} units before the start of a \ty{Range}
			& $f : \ty{R},\ty{D} \rightarrow \ty{R}$ \\
		\hline
		% -- Arity 1
		%(seq move)
		\texttt{moveLeft1}
			& Move the origin of a \ty{sequence} left by 1
			& $f : \ty{S} \rightarrow \ty{S}$ \\
		\texttt{moveRight1}
			& Move the origin of a sequence right by 1
			& $f : \ty{S} \rightarrow \ty{S}$ \\
		\hline
		%(n^th element)
		\texttt{\th{$n$} $x$ of $y$}
			& Take the \th{$n$} \ty{Sequence} in $y$ (Day of Week, \textit{etc})
			& $f : \ty{Number} \rightarrow \ty{S}$ \\
		\hline
		%(fuzzify)
		\texttt{approximate}
			& Make a \ty{Duration} approximate
			& $f : \ty{D} \rightarrow \ty{D}$ \\
		\hline
	\end{tabular}
	\caption{
		The functional preterminals of the grammar; \ty{R}, \ty{S}, and \ty{D}
			denote \ty{Range}s \ty{Sequence}s and \ty{Duration}s respectively.
		The name, a brief description, and the type signature of the function
			(as used in parsing) are given.
		Described in more detail in \refsec{repr-rules}, the functions are most
			easily interpreted as operations on either an interval or sequence.
	}
	\label{tab:function}
	\end{center}
\end{table*}

% -- Intro
We define a set of preterminals, specifying their eventual
	\textit{type}, as well as the temporal \textit{instance} it
	produces when its function is evaluated on the word it generates
	(e.g., $f(\tp{day}) = \te{Day}$).
A distinction is made in our description between entities with 
	content roles versus entities with a functional role.

% -- Content
The first -- consisting of \ty{Range}s, \ty{Sequence}s, and \ty{Duration}s --
	are listed in \reftab{content}.
A total of 62 such preterminals are defined in the implemented system
	corresponding to primitive entities often appearing in newswire,
	although this list is easily adaptable to fit other domains.
It should be noted that the expressions, represented in \te{Typewriter},
	have no a prior association with word,s denoted by \tp{italics};
	this correspondence must be learned.
Furthermore, entities which are subject to interpretation -- for example
	\te{Quarter} or \te{Season} -- are given a concrete interpretation.
The \th{$n$} quarter is defined by evenly splitting a year into four;
	the seasons are defined in the same way but with winter beginning in December.

% -- Functional
The functional entities are described in \reftab{function}, and correspond to
	the \ty{Function} type.
The majority of these mirror generic operations on intervals on a timeline,
	or manipulations of a sequence.
Notably, like intervals, times can be moved (\tp{3 weeks ago}) or
	their size changed (\tp{the first two days of the month}), or
	a new interval can be started from one of the endpoints
	(\tp{the last 2 days}).
Additionally, a sequence can be modified by shifting its origin
	(\tp{last Friday}), 
	or taking the \th{$n$} element of the sequence within some bound
	(\tp{fourth Sunday in November}).

% -- Other
The lexical entry for the \ty{Nil} type is tagged with the word it
	generates, producing entries such as \ty{Nil(a)}, \ty{Nil(November)}, etc.
The lexical entry for the \ty{Number} type is further parameterized by
	the order of magnitude and ordinality of the number; e.g.,
	\th{27} becomes \ty{Number($10^1$,ordinal)}.


%%%%%%%%%%%%%%%%%%%%%%%%%%%%%%%%%%%%%%%%%%%%%%%%%%%%%%%%%%%%%%%%%%%%%%%%%%%%%%%%
% RULES
%%%%%%%%%%%%%%%%%%%%%%%%%%%%%%%%%%%%%%%%%%%%%%%%%%%%%%%%%%%%%%%%%%%%%%%%%%%%%%%%
\Subsection{repr-rules}{Combination Rules}
% -- Intro
As mentioned earlier, our grammar defines both combination rules
	over types (in $\sV$) as well as a method for combining temporal
	instances (in $W_v \in sW$).
This method is either a function application of one of the functions in
	\reftab{function}, a function which is implicit in the text
	(intersection and multiplication), or an identity operation (for \ty{Nil}s).
These cases are detailed below:

\begin{itemize}
	\item Function application, e.g., \tp{last week}.
		We apply (or partially apply) a function to an argument
			on either the left or the right -- $f(x,y) \odot x$ or $x \odot f(x,y)$.
		Furthermore, for functions of arity 2 taking a \ty{Range} as an argument,
			we define a rule treating it as a unary function with the reference time
			taking the place of the second argument.
	%(intersection)
	\item Intersecting two ranges or sequences, e.g., \tp{November \th{27}}.
		The intersect function treats both arguments as intervals, and will return
			an interval (\ty{Range} or \ty{Sequence}) corresponding to the overlap
			between the two.
			\footnote{
				In the case of complex sequences 
					(e.g., \tp{Friday the \th{13}}) an A$^{*}$
					search is performed to find overlapping ranges in the two sequences;
					the origin $r_s(0)$ is updated to refer to the closest such
					match to the reference time.
		}.
	%(mult)
	\item Multiplying a \ty{Number} with a \ty{Duration}, e.g., \tp{5 weeks}.
	%(nil)
	\item Combining a non-\ty{Nil} and \ty{Nil} element with no change to the
			temporal expression, e.g., \tp{a week}.
		The lexicalization of the \ty{Nil} type allows the algorithm
			to take hints from these supporting words.
\end{itemize}

% -- Segue
We proceed to describe learning the parameters of this grammar.

