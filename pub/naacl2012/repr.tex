\Section{repr}{Representation}
A temporal representation is constructed; the type system of the representation
	is described in \refsec{repr-types} while the grammar is outlined in
	\refsec{repr-compositional} and described in detail in 
	\refsecs{repr-lexicon}{repr-rules}

%%%%%%%%%%%%%%%%%%%%%%%%%%%%%%%%%%%%%%%%%%%%%%%%%%%%%%%%%%%%%%%%%%%%%%%%%%%%%%%%
% TEMPORAL EXPRESSIONS
%%%%%%%%%%%%%%%%%%%%%%%%%%%%%%%%%%%%%%%%%%%%%%%%%%%%%%%%%%%%%%%%%%%%%%%%%%%%%%%%
\Subsection{repr-types}{Temporal Expression Types}
% -- Intro to Types
Instances of temporal expressions are grouped into one of the following types:
% -- Range
\paragraph{Range}
%(definition)
A period between two dates (or times).
%(examples)
This includes entities such as \te{Today}, \te{1987}, or
	\te{Now}.
We denote a range by the variable \range.
We maintain a consistent interval-based theory of time
	\cite{key:1981allen-temporal} and instances as intervals with
	zero span.

% -- Sequence 
\paragraph{Sequence}
%(definition)
A sequence of \ty{Range}s, not necessarily occurring at regular intervals.
%(examples)
This includes entities such as \te{Friday}, \te{November \th{27}}, or
	\te{last May}.
%(anatomy)
A given sequence is in reality a tuple of three elements
	$\seq=(\seqelem,\seqjump,\seqcanon)$:
\begin{enumerate}
	\setlength{\itemsep}{-5pt} 
	\item $\seqelem(i)$: 
		The \th{i} \ty{Range} of a sequence.
		This is a \ty{Range} by type, and is the term which the sequence ultimately
			grounds to, for some $i$.
		In the case of the sequence \te{Friday}, $\seqelem(0)$ corresponds to
			\tp{the Friday in the current week}; 
			$\seqelem(1)$ is the Friday in the following week, \textit{et cetera}.
	\item \seqjump:
		The \textit{distance} between two elements in the sequence -- approximated
			if this distance is not constant.
		In the case of \te{Friday}, this distance would be a \te{Week}.
	\item \seqcanon:
		The \textit{containing unit} of an element of a sequence.
		For example, \seqcanonX{Friday} would be the \ty{Range}
			corresponding to \tp{the current week}.
		The sequence index $i \in \BZ$ is defined relative to $\seqelem(0)$ --
			the range in the same containing unit as the reference time.
\end{enumerate}

%(definition of reference time)
We define the \textit{reference time} $t$ \cite{key:1947reichenback-temporal}
	to be the instant when the phrase 
	is uttered; for the \tempeval\ corpus, we assume that this is the 
	publication time of the article.

%(sequence as distribution)
The \ty{Sequence} type warrants special attention, as the pragmatic ambiguity 
	when describing temporal expressions resides in the interpretation of these
	terms.
A \ty{Sequence} denotes not only a sequence of possible
	\ty{Range}s, but also a distribution over these ranges.
This distribution -- conditioned on $t$ --
	is defined as a Gaussian with a mean and variance defined according
	to the distance \seqjump\.
The input to the Gaussian is a function $\delta_{s,t}$:
	the distance between the reference time 
	and the element in question.
% -- Gaussian Equation
\begin{equation}
\label{eqn:gaussian}
	P_t(i) = 
	\int_{i-0.5}^{i+0.5}
		\frac{
			\exp\left[-\frac{
					(\frac{\delta_{s,t}(i)-\delta_{s,t}(0)}{\seqjump} - \mu)^2
				}{
					2\sigma^2
				}
			\right]
		}{
			2\pi\sigma^2
		}
\end{equation}
% -- Gaussian Plot
\begin{figure}[t]
\begin{center}
	\resizebox{1.1\hsize}{!}{
		\input fig/gauss.tex
	}
	\caption{
		\label{fig:distribution}
		An illustration of a temporal distribution.
		The reference time is labeled as time $t$ between \te{Nov 20} and 
			\te{Nov 27}; the probability that this sequence is referring to
			\te{Nov 20} is integral of the marked area.
		Note that shifting the reference time towards \te{Nov 20} would increase
			this probability.
	}
\end{center}
\end{figure}
\reffig{distribution} shows an example of such a distribution.

%(global scope)
The parameters $\mu$ and $\sigma$ in Equation \refeqn{gaussian} are shared
	among all sequence instances; the normalization
	with respect to \seqjump\ ensures that the distribution is applicable
	to any particular sequence.
	
% -- Duration
\paragraph{Duration}
%(definition)
A period of time.
%(examples)
This includes entities like \te{Week}, \te{Month}, and \te{7 days}.
%(etc.)
We denote a duration with the variable \dur.

%(fuzzy)
We define a special case of the \ty{Duration} type to represent 
	\textit{approximate} durations.
These are used to represent expressions such as \tp{a few years} or
	\tp{some days}.
Semantically, they are equivalent to any other approximate
	duration with the same unit and distinct from any other \ty{Duration}.
This is consistent with the representation of such entities in the
	\texttt{Timex} format \cite{key:2003pustejovsky-timeml}.

% -- Functions
\paragraph{Function}
%(definition)
A function of arity less than or equal to two representing some
	general modification to one of the above types.
%(examples)
This captures semantic entities such as those implied in
	\tp{last $x$}, \tp{the third $x$ [of $y$]}, or \tp{$x$ days ago}.
%(etc.)
The particular functions and their application are discussed in 
	\reftab{function}.


% -- Other
\paragraph{Other Types}
Two other types bear auxiliary roles in representing temporal expressions,
	though they are not directly temporal concepts.
In the grammar, these appear as preterminals only.

The first of these types is \ty{Number} -- denoting a number without
	any temporal meaning attached.
This comes into play representing expressions such as \tp{2 weeks}.
The other type is the \ty{Nil} type -- denoting terms which are not
	directly contributing to the semantic meaning of the expression.
This is intended for words such as \tp{a} or \tp{the}, which serve as cues
	without bearing temporal content themselves.
The \ty{Nil} type is lexicalized with the word it generates; more details
	are given in \refsecs{repr-lexicon}{repr-rules}.


\paragraph{Omitted Phenomena}
The representation described is a simplification of the complexities of
	time.
Notably, a body of work has focused on reasoning about events or states
	relative to temporal expressions -- 
	\newcite{key:1988moens-temporal} describes temporal expressions relating to
	changes of state; 
	\newcite{key:2010condoravdi-temporal} explores NPI licensing in temporal
	expressions.
Furthermore, vague times represent a notable chunk of temporal expressions
	uttered.
In contrast, NLP evaluations have generally not included such vague 
	time expressions.

%%%%%%%%%%%%%%%%%%%%%%%%%%%%%%%%%%%%%%%%%%%%%%%%%%%%%%%%%%%%%%%%%%%%%%%%%%%%%%%%
% TIME AS COMPOSITIONAL PARSING
%%%%%%%%%%%%%%%%%%%%%%%%%%%%%%%%%%%%%%%%%%%%%%%%%%%%%%%%%%%%%%%%%%%%%%%%%%%%%%%%
\Subsection{repr-compositional}{Synchronous PCFG}
% -- Time is compositional
Our approach builds on the assumption that the natural language description
	of time is compositional in nature.
Unlike many aspects of language  -- which are often not compositional or
	idiomatic -- each word attached to a temporal phrase is 
	usually compositionally modifying the meaning of the phrase.
To demonstrate, we consider the expression \tp{last Friday the \th{13}}.
Decomposing the example, we can construct a meaning by 
	applying the modifiers \tp{last} to \tp{Friday} -- 
	creating the previous \te{Friday};
	and \tp{the \th{13}} -- creating the previous \te{Friday} that is
	also the \th{13} of the month.

% -- Focus on time; elsewhere applicable
While we focus entirely on temporal phrases and their meaning, this
	compositional assumption is reasonable for a number of other domains as 
	well -- for example describing locations (e.g., the 
	\dataset{GeoQueries} dataset) or directions.

% -- Synchronous Grammar
\Fig{fig/grammar}{0.45}{grammar}{
	The synchronous grammar -- (a) describes the CFG parse of the
		temporal \textit{types}; (b) describes the corresponding combination
		of the temporal \textit{instances}.
	The phrase is chosen to demonstrate ambiguity both in the parse
		(\tp{last Friday} could equally well be a constituent) and in the
		interpretation (imagine the phrase uttered on Friday the \th{14}).
}
%(introduction)
We construct a paradigm for parsing temporal phrases based on a Synchronous
	PCFG defined over the temporal \textit{instances} and their
	corresponding \textit{types}, similar in some respects but more
	constrained than those used in machine translation 
	(e.g., \newcite{key:2001yamada-syntaxmt}).
This is in contrast to common approaches, such as making use of CCG 
	grammars 
	\cite{key:2000steedman-ccg,key:2004bos-ccg,2011kwiatkowski-semantics}
	giving us more flexibility in the composition rules.
\reffig{grammar} shows an example of the grammar.

%(formal definition)
Formally, we define a Synchronous PCFG 
	$G = \left(\Sigma,S,\sV,\sW,\sR,P\right)$.
The alphabet $\Sigma$ and start symbol $S$ retain their usual interpretations.

%(nonterminals)
We define a set $\sV$ to be the set of types as described in
	\refsec{repr-types} -- these act as our nonterminals.
For each $v \in \sV$ we define an (unbounded) set $W$ corresponding to the 
	possible instances of type $v$.
Concretely, if $v=\ty{Sequence}$, our set $W_v \in \sW$ could contain elements
	corresponding to \tp{Friday}, \tp{last Friday}, \tp{Nov. \th{27}},
	\textit{et cetera}.
The synchronous grammar is constructed over pairs $(v,w \in W_v)$.

%(rules)
A rule $R \in \sR$ is defined as a pair 
	$R = \left(v_i \rightarrow v_jv_k, 
		f : (W_{v_j},W_{v_k}) \rightarrow W_{v_i}\right)$
The first term is a conventional PCFG rule over the types
	$\sV$.
In a conventional synchronous PCFG, the second term would be the corresponding
	combination rule over our nonterminals $\sW$.
In our case, this term is a deterministic function mapping the
	constituent expressions to the parent expression.
That is, we are learning a rule-to-rule translation 
	\cite{key:1976bach-semantics}.
%This both handles the unbounded domain of temporal expressions,
%	as well as constrains the formalism.
Note that the definition above is trivially adapted for the case of unary
	rules.
The application of these rules is described in more detail in 
	\refsec{repr-rules}.

%(probabilities)
The last term in our grammar formalism denotes the rule probabilities $P$.
In line with the usual interpretation, this defines a probability
	of applying a particular rule $r \in R$.
Importantly, note that the distribution over possible groundings
	of a temporal expression are not included in the grammar
	formalism.
The learning of these probabilities is detailed in \refsec{learn}.

%%(motivation)
%We defined a paradigm for expressing how natural language composes
%	temporal phrases while accounting for the sparsity inherent in composing
%	individual expressions -- as well as helping ensure that the resulting
%	expressions are semantically valid.
%In parallel, a particular temporal expression is parsed, which can then
%	be grounded by a reference time to provide the final temporal expression.
%We continue to describe the preterminals of our grammar and the introduction
%	of our synchronous terms.


%%%%%%%%%%%%%%%%%%%%%%%%%%%%%%%%%%%%%%%%%%%%%%%%%%%%%%%%%%%%%%%%%%%%%%%%%%%%%%%%
% LEXICON
%%%%%%%%%%%%%%%%%%%%%%%%%%%%%%%%%%%%%%%%%%%%%%%%%%%%%%%%%%%%%%%%%%%%%%%%%%%%%%%%
\Subsection{repr-lexicon}{Preterminals}
% -- Static Lexicon
\begin{table}[tb]
	\begin{center}
	\begin{tabular}{|l|l|}
		\hline
		\textbf{Type} & \textbf{Instances} \\
		\hline
		\hline
		% -- Ranges
		\ty{Range} &
			\te{Past}, \te{Future}, \te{Yesterday}, \\
			& \te{Tomorrow}, \te{Today}, \te{Reference}, \\
			& \te{Year($n$)}, \te{Century($n$)} \\
		\hline
		% -- Sequences
		\ty{Sequence} 
			%(named sequences)
			& \te{Morning}, \te{Afternoon}, $\dots$ \\
			& \te{Monday}, \te{Tuesday}, $\dots$ \\
			& \te{January}, \te{February}, $\dots$ \\
			& \te{Winter}, \te{Spring}, $\dots$ \\
			%(from number)
			& \te{MinuteOfHour($0 \dots 59$)}, \\
			& \te{HourOfDay($0 \dots 23$)}, \\
			& \te{DayOfMonth($1 \dots 31$)}, \\
			& \te{MonthOfYear($1 \dots 12$)}, \\
			& \te{YearOfCentury($0 \dots 99$)}, \\
			& \te{DecadeOfCentury($0 \dots 9$)}, \\
			& \te{YearOfDecade($0 \dots 9$)}, \\
			%(durational)
			& \te{EveryDay}, \te{EveryWeek}, \\
			& \te{EveryMonth}, \te{EveryQuarter}, \\
			& \te{EverySeason} \\
		\hline
		% -- Durations
		\ty{Duration}
			& \te{Second}, \te{Minute}, \te{Hour}, \\
			& \te{Day}, \te{Week}, \te{Month}, \te{Quarter}, \\
			& \te{Year}, \te{Decade}, \te{Century} \\
		\hline
	\end{tabular}
	\caption{
		The content-bearing preterminals of the grammar, arranged by their
			types.
		Note that the elements on the right denote the semantic meaning of the
			element and have no relation to the phrase given;
			furthermore, lines ending with ellipses denote more elements of the
			same type which have been omitted for brevity.
	}
	\label{tab:content}
	\end{center}
\end{table}

% -- Function Lexicon
\begin{table*}[t]
	\begin{center}
	\begin{tabular}{|l|l|l|}
		\hline
		\textbf{Function} & \textbf{Description} & \textbf{Signature(s)} \\
		\hline
		\hline
		% -- Arity 2
		%(shift left)
		\texttt{shiftLeft}
			& Shift a \ty{Range} or \ty{Sequence} left by a \ty{Duration}
			& $f : \ty{S},\ty{D} \rightarrow \ty{S}$; ~~
			  $f : \ty{R},\ty{D} \rightarrow \ty{R}$ \\
		\texttt{shiftRight}
			& Shift a \ty{Range} or \ty{Sequence} right by a \ty{Duration}
			& $f : \ty{S},\ty{D} \rightarrow \ty{S}$; ~~
			  $f : \ty{R},\ty{D} \rightarrow \ty{R}$ \\
		\hline
		%(shrink begin)
		\texttt{shrinkBegin}
			& Take the first \ty{Duration} of a \ty{Range}/\ty{Sequence}
			& $f : \ty{S},\ty{D} \rightarrow \ty{S}$; ~~
			  $f : \ty{R},\ty{D} \rightarrow \ty{R}$ \\
		\texttt{shrinkEnd}
			& Take the last \ty{Duration} of a \ty{Range}/\ty{Sequence}
			& $f : \ty{S},\ty{D} \rightarrow \ty{S}$; ~~
			  $f : \ty{R},\ty{D} \rightarrow \ty{R}$ \\
		\hline
		\texttt{catLeft}
			& Take \ty{Duration} units after the end of a \ty{Range}
			& $f : \ty{R},\ty{D} \rightarrow \ty{R}$ \\
		\texttt{catRight}
			& Take \ty{Duration} units before the start of a \ty{Range}
			& $f : \ty{R},\ty{D} \rightarrow \ty{R}$ \\
		\hline
		% -- Arity 1
		%(seq move)
		\texttt{moveLeft1}
			& Move the \textit{zero index} of a \ty{sequence} left by 1
			& $f : \ty{S} \rightarrow \ty{S}$ \\
		\texttt{moveRight1}
			& Move the \textit{zero index} of a sequence right by 1
			& $f : \ty{S} \rightarrow \ty{S}$ \\
		\hline
		%(n^th element)
		\texttt{\th{$n$} $x$ of $y$}
			& Take the \th{$n$} \ty{Sequence} in $y$ (Day of Week, \textit{etc})
			& $f : \ty{Number} \rightarrow \ty{S}$ \\
		\hline
		%(fuzzify)
		\texttt{approximate}
			& Make a \ty{Duration} approximate
			& $f : \ty{D} \rightarrow \ty{D}$ \\
		\hline
	\end{tabular}
	\caption{
		The functional preterminals of the grammar; \ty{R}, \ty{S}, and \ty{D}
			denote \ty{Range}s \ty{Sequence}s and \ty{Duration}s respectively.
		The name, a brief description, and the type signature of the function
			(as used in parsing) are given.
		Described in more detail in \refsec{repr-rules}, the functions generally 
			apply generic interval or sequence modifications.
	}
	\label{tab:function}
	\end{center}
\end{table*}

% -- Intro
We define a set of preterminals, consisting of pairs of 
	\textit{types} and \textit{instances}	of temporal expressions.
A distinction is made in our description between entities with 
	content roles versus entities with a functional role.

% -- Content
The first -- consisting of \ty{Range}s, \ty{Sequence}s, and \ty{Duration}s --
	are listed in \reftab{content}.
A total of 57 such preterminals are defined in the implemented system,
	although this list is easily adaptable to fit specialized domains.
It should be noted that, although the expressions are described in English,
	their meaning is purely semantic.
Furthermore, entities which are subject to interpretation -- for example
	\te{Quarter} or \te{Season} -- are given a concrete interpretation.
The \th{$n$} quarter is defined by evenly splitting a year into four;
	the seasons are defined in the same way but with winter beginning in December.

% -- Functional
The functional entities are described in \reftab{function}, and correspond to
	the \ty{Function} type.
The majority of these mirror generic operations on intervals on a timeline,
	or manipulations of a sequence.
Notably, like intervals, times can be moved (\tp{3 weeks ago}) or
	their size changed (\tp{the first two days of the month}), or
	a new interval can be started from one of the endpoints
	(\tp{the last 2 days}).
Additionally, a sequence can be modified by shifting its \textit{zero index}
	(\tp{last Friday}), 
	or taking the \th{$n$} element of the sequence within some bound
	(\tp{fourth Sunday in November}).

% -- Other
The lexical entries for the \ty{Nil} and \ty{Number} types are straightforward,
	though they play an important role in parsing, as described later.


%%%%%%%%%%%%%%%%%%%%%%%%%%%%%%%%%%%%%%%%%%%%%%%%%%%%%%%%%%%%%%%%%%%%%%%%%%%%%%%%
% RULES
%%%%%%%%%%%%%%%%%%%%%%%%%%%%%%%%%%%%%%%%%%%%%%%%%%%%%%%%%%%%%%%%%%%%%%%%%%%%%%%%
\Subsection{repr-rules}{Grammar}
% -- Intro
The synchronous PCFG grammar consists of a standard CFG rule over temporal
	types (in $\sV$) paired with a deterministic procedure for combining instances
	of those types (in $W_v \in \sW$).
This deterministic procedure is in effect a function application,
	though in particular it breaks down into three operations:

% -- Operators
\begin{itemize}
	\setlength{\itemsep}{-5pt} 
	%(function application)
	\item Applying (or partially applying) a function with one of its
			arguments.
		A separate rule exists for applying on the left versus on the right --
		that is, for composing $f(x,y) \odot x$ versus $x \odot f(x,y)$.
	%(intersection)
	\item Intersecting two ranges or sequences.
		This roughly equates to applying a special \textit{intersect} function
			without requiring that it be explicit in the temporal phrase.
		The intersect function treats both arguments as intervals, and will return
			the overlap between the intervals\footnote{
				In the case of sequences (e.g., \te{\th{13}} intersect \te{May}) a
					search is performed to find overlapping ranges in the two sequences;
					the \textit{zero index} is updated to refer to the closest such
					match to the reference time.
				In cases where this search cannot be solved analytically
					(e.g., \te{\th{13}} intersect \te{Friday}), an A$^{*}$
					search is used to find an overlapping pair of ranges
					in time logarithmic with respect to
					the distance in miliseconds to the match.
		}.

	%(noop)
	\item Combining a non-\ty{Nil} element with \ty{Nil} with no change to the
			temporal expression.
		The lexicalization of the \ty{Nil} type allows the algorithm
			to take hints from these supporting words.
\end{itemize}

% -- Segue
We have defined a framework for representing and parsing temporal phrases
	based on a synchronous PCFG.
We proceed to describe learning the parameters of this grammar.

