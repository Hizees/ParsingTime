\Section{repr}{Representation}

%%%%%%%%%%%%%%%%%%%%%%%%%%%%%%%%%%%%%%%%%%%%%%%%%%%%%%%%%%%%%%%%%%%%%%%%%%%%%%%%
% TIME AS COMPOSITIONAL PARSING
%%%%%%%%%%%%%%%%%%%%%%%%%%%%%%%%%%%%%%%%%%%%%%%%%%%%%%%%%%%%%%%%%%%%%%%%%%%%%%%%
\Subsection{repr-compositional}{Time As Compositional Parsing}
\todo{Compositional semantics citations}
% -- Time is compositional
We make the simplifying assumption that the natural language description
	of time is compositional in nature.
Unlike many aspects of language  -- which are either not compositional or
	riddled with idioms -- as a temporal phrase increases in length it is
	usually compositionally modifying the meaning of a shorter segment of itself.
To demonstrate and introduce a running example, we consider the expression
	\tp{last Friday the \th{13}}.
Decomposing the example, we can construct a meaning by taking \tp{Friday},
	and applying the modifiers \tp{last} -- taking the previous \te{Friday};
	and \tp{the \th{13}} -- taking the previous \te{Friday} that is
	also the \th{13} of the month.

% -- Focus on time; elsewhere applicable
While we focus entirely on temporal phrases and their meaning, this
	compositional assumption is reasonable for a number of other domains as 
	well -- for example describing locations or directions (e.g. the 
	\dataset{GeoQueries} dataset \needcite).

% -- Synchronous Grammar
We construct a paradigm for parsing temporal phrases based on a synchronous
	PCFG defining probabilities over the combinations of \textit{types}
	of temporal expressions.
\needfig shows an example of such a grammar.
This defines a paradigm for expressing how natural language composes
	temporal phrases without accounting for the sparsity inherent in composing
	individal expressions -- as well as helps ensure that the resulting
	expressions are semantically valid.
In parallel, a particular temporal expression is parsed, which can then
	be grounded by a reference time to provide the final temporal expression.

% -- Table of Contents
The representation of the semantic entries in the two aspects of this grammar
	are described in \refsec{repr-background}.
The lexical entries (preterminals) are described in \refsec{repr-lexicon};
	the combination rules of the grammars are described in \refsec{repr-rules}.

%%%%%%%%%%%%%%%%%%%%%%%%%%%%%%%%%%%%%%%%%%%%%%%%%%%%%%%%%%%%%%%%%%%%%%%%%%%%%%%%
% TEMPORAL EXPRESSIONS
%%%%%%%%%%%%%%%%%%%%%%%%%%%%%%%%%%%%%%%%%%%%%%%%%%%%%%%%%%%%%%%%%%%%%%%%%%%%%%%%
\Subsection{repr-background}{Temporal Expressions}
% -- Intro to Types
We define the representation of temporal entities to be used in parsing.
We assume that every temporal expression falls into one of the following
	categories:
% -- Range
\paragraph{Range}
%(definition)
A period between two dates (or date$+$times).
%(examples)
This includes entities such as \te{Today}, \te{1987}, or
	\te{Now}.
We denote a range by the variable \range.

%(point vs. interval based)
The third of these is worth a particular mention:
	unlike the previous examples, the start and end of the range are the same.
That is, \te{Today} spans a day, \te{1987} spans a year; \te{Now}
	spans no time.
A point-based representation of time -- such as those described in
	certain works \needcite -- would dictate this as a distinct type.
To simplify our grammar and representation, we adopt an interval-based
	representation \needcite.

% -- Sequence 
\paragraph{Sequence}
%(definition)
A sequence of \ty{Range}s.
%(examples)
This includes entities such as \te{Friday}, \te{November \th{27}}, or
	\te{last May}.
%(anatomy)
A given sequence is in reality a tuple of three elements
	$\seq=(\seqelem,\seqjump,\seqcanon)$:
\begin{enumerate}
	\item $\seqelem(i)$: 
		The \th{i} element of a sequence.
		This is a \ty{Range} by type, and is the type returned when a sequence is
			asked to ground.A
		In the case of the sequence \te{Friday}, $\seqelem(0)$ corresponds to
			\tp{the Friday in the current week}; 
			$\seqelem(1)$ is the Friday in the following week, \textit{et cetera}.
	\item \seqjump:
		The \textit{distance} between two elements in the sequence.
		In the case of \te{Friday}, this distance would be a \te{Week}.
	\item \seqcanon:
		The \textit{containing unit} of an element of a sequence.
		For example, \seqcanonX{Friday} would be the \ty{Range}
			corresponding to \tp{the current week}.
		In the case of \tp{November}, or even \tp{November \th{27}}. this 
			would correspond to the current year.
\end{enumerate}

%(definition of relative index)
The index $i \in \BZ$ is defined as the element of the sequence
	that is within the containing unit \seqcanon of the sequence in which the 
	reference time also lies.
This is consistent with our definition of \seqelem.

%(sequence as distribution)
This type warrants special attention, as the pragmatic ambiguity when
	describing temporal expressions resides in the interpretation of these
	terms.
Therefore, a \ty{Sequence} denotes not only a sequence of possible
	\ty{Range}s, but also a distribution over these ranges.
We define this distribution -- conditioned on a reference time $t$ --
	as a Gaussian with a mean and variance defined according
	to the distance \seqjump\ between two elements in a given sequence $s$.
The input to the Gaussian is the index $i$ of the element in the sequence;
	and, the distance between the reference time and the beginning
	of the element in question $\delta_{s,t}(i)$:
\begin{equation}
\label{eqn:gaussian}
	P_t(i) = 
	\int_{i-0.5}^{i+0.5}
		\frac{1}{2\pi\sigma^2} 
			\exp\left[-\frac{
					(\frac{\delta_{s,t}(i)}{\seqjump} - \mu)^2
				}{
					2\sigma^2
				}
			\right]
\end{equation}
\needfig shows an example of such a distribution.

%(global scope)
The parameters $\mu$ and $\sigma$ in Equation \refeqn{gaussian} are shared
	among all particular sequence instances; the normalization
	with respect to \seqjump\ ensures that the distribution is applicable
	to any particular sequence.
For example, the distance between a Wednesday to that week's \te{Friday}
	would be $2/7$, counted with respect to a week; 
	the distance from April to that year's \te{November} would be a 
	comparable $7/12$ (approximately), counted with respect to a month.
	
% -- Duration
\paragraph{Duration}
%(definition)
A period of time -- effectively a number of milliseconds.
%(examples)
This includes entities like \te{Week}, \te{Month}, and \te{7 days}.
%(etc)
We denote a duration with the variable \dur.

%(fuzzy)
We define a special case of the \ty{Duration} type to represent 
	\textit{approximate} durations.
These are used to represent expressions such as \tp{a few years} or
	\tp{some days}.
Semantically, these are represented by a single unit of duration with the
	quantity omitted, and are taken to be equivalent to any other approximate
	duration with the same unit and distinct from any other \ty{Duration}.
This is consistent with the representation of such entities in the
	\texttt{Timex} format.

% -- Functions
\paragraph{Functions}
\dome

% -- Other
\paragraph{Other Types}
In addition to the types described above, there are two other types which
	have relevance in the parsing as described in \refsec{learn}.
Both of these types are applicable as preterminals only.

The first of these types is \ty{Number} -- denoting a number without
	any semantic meaning attached yet.
This comes into play with expressions such as \tp{2 weeks}, in which
	the number $2$ has no temporal meaning on its own.

The other type are the \ty{Nil} types -- denoting terms which are not
	directly contributing to the semantic meaning of the expression.
These are intended for words such as \tp{a} or \tp{the}, which serve as cues
	informing the resulting temporal expression without being meaningful
	in themselves.
The \ty{Nil} type is lexicalized with the word it generates; more details
	are given in \refsecs{repr-lexicon}{repr-rules}.
Naturally, our representation is incomplete in capturing all possible
	temporal phenomena; a subset of these omissions are given below.


\paragraph{Omitted Phenomena}
The temporal representation described above is a simplification of the
	possible phenomena that occur in time.
Among these, \todo{describe and cite stuff here}.

Practically, our representation is unable to represent approximate ranges;
	for example, \tp{a year in the 90's} or \tp{some evening}.
These are valid in the \texttt{Timex} specification; therefore, the decision
	to omit these for simplicity comes at the expense of a small performance hit.

We proceed to describe the grammar used to manipulate the types and specific
	instances of these types.
\refsec{repr-lexicon} describes the instances of the types described
	above; \refsec{repr-rules} describes the grammar rules of the types,
	and their correpsonding semantic impact on the instances.


%%%%%%%%%%%%%%%%%%%%%%%%%%%%%%%%%%%%%%%%%%%%%%%%%%%%%%%%%%%%%%%%%%%%%%%%%%%%%%%%
% LEXICON
%%%%%%%%%%%%%%%%%%%%%%%%%%%%%%%%%%%%%%%%%%%%%%%%%%%%%%%%%%%%%%%%%%%%%%%%%%%%%%%%
\Subsection{repr-lexicon}{Lexical Entries}

%%%%%%%%%%%%%%%%%%%%%%%%%%%%%%%%%%%%%%%%%%%%%%%%%%%%%%%%%%%%%%%%%%%%%%%%%%%%%%%%
% RULES
%%%%%%%%%%%%%%%%%%%%%%%%%%%%%%%%%%%%%%%%%%%%%%%%%%%%%%%%%%%%%%%%%%%%%%%%%%%%%%%%
\Subsection{repr-rules}{Rule Applications}

The words in a temporal expression are treated as units of meaning that
	combine compositionally.

Our lexicon contains two types of items
\begin{itemize}
	\item A large set of fundamental concepts denoting entities of time
	\item A small number of general function applications on these concepts
\end{itemize}
\dome
