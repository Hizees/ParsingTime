\Section{result}{Evaluation}
We evaluate our system against current state-of-the art systems for temporal
	resolution on the english portion of the \tempeval\ Task A dataset.
We describe the dataset in \refsec{result-dataset}, and detail the previous
	work and our results in \refsec{result-scores}.

\Subsection{result-dataset}{Dataset}
The \tempeval\ dataset contains 162 documents and 1052 temporal 
	phrases in the training set,
	and an additional 20 documents and 156 phrases in the evaluation set.
Each temporal phrase was annotated as a \timex\footnote{
		see \url{http://www.timeml.org} for details on the TimeML format and
		\timex\ tag.
	}
	tag around an adverbial or prepositional phrase

Our System was trained on 142 documents, with the remaining 20 held out as
	a development set.
Final reported results make use of the full 162 documents in the training
	set.




\Subsection{result-scores}{Results}
% -- Result Table
\begin{table}
	\begin{tabular}{|l|c|c|c|c|}
		\hline
		       & \multicolumn{2}{c|}{Type Accuracy} & \multicolumn{2}{c|}{Value Accuracy} \\
		System & Train & Test  & Train & Test\\
		\hline
		\hline
		\sys{GUTime}     & 0.72          & 0.79          & 0.45          & 0.41 \\
		\sys{SUTime}     & 0.85          & 0.90          & \textbf{0.69} & 0.67 \\
		\sys{HeidelTime} & 0.80          & 0.85          & 0.67          & \textbf{0.71} \\
		\hline
		\sys{OurSystem}  & \textbf{0.90} & \textbf{0.93} & \textbf{0.69} & \textbf{0.71} \\
		\hline
	\end{tabular}
	%(caption)
	\caption{
		\tempeval\ Attribute scores for our system and three previous systems.
		The scores are calculated using gold extents, forcing a guessed
		interpretation for each parse.
		The results of \sys{GUTime} have been reformatted to conform to the
			\tempeval\ output format.
		\label{tab:results}
	}
\end{table}

% -- Evaluation task
In the \tempeval\ A Task, system performance is evaluated on 
	detection and resolution of temporal expressions.
Since we perform only the second of these, we evaluate our system
	assuming gold detection.
\todo{PR curves maybe?}
%Thus, only the \timex\ attribute evaluation is compared against.

% -- Scoring Tweak
Similarly, the original \tempeval\ scoring scheme gave a precision 
	and recall for detection, and an accuracy for only the temporal expressions 
	attempted.
Since our system will output a guess for every expression, we compare to 
	previous system scores when constrained to make a prediction on every
	example; if no guess is made, the output is considered incorrect.
This in general yields lower results, as the system is not allowed to
	abstain on expressions it does not recognize.
% -- Reference Table
Results are summarized in \reftab{results}

% -- Systems
We compare to three previous rule-based systems.
%(GUTime)
\sys{GUTime} \cite{key:2000mani-temporal} presents an older but widely
	used baseline.
Due to discrepancies in the output format of \sys{GUTime} compared to the 
	format used in the \tempeval\ task, the output was heuristically patched
	and manually checked to conform to the expected format.
%(SuTime)
More recently, \sys{SUTime} \textit{(under submission 2011)} 
	provides a much stronger comparison.
%(HeidelTime)
We also compare to \sys{HeidelTime} \cite{key:2010strotgen-temporal}, 
	which represents the state-of-the-art system at the \tempeval\ task.

% -- Our Results
Our system performs well above the \sys{GUTime} baseline and is competitive
	with both of the more recent systems.
\todo{examples}



