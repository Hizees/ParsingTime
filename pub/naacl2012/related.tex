\Section{related}{Related Work}
% -- Semantic Parsing Comparison
Our approach draws inspiration from a large body of work
	on parsing expressions into a logical form.
The latent parse parallels the formal semantics in previous work,
	e.g., Montague semantics.
Like these representations, a parse -- in conjunction with
	the reference time -- completely defines a set of
	matching entities, in this case the grounded time.
The matching times can be thought of as analogous to the entities
	in a logical model which satisfy a given expression.

%%(logic analogy)
%The complexity of parsing temporal expressions as nested function
%	applications is similar to that of parsing to logical forms.
%Whereas the latter involves identifying the truth of statements
%	in a logical \textit{model} (in most cases, a database),
%	our model filters and weighs the probability of grounded times
%	which are consistent with the given temporal expression.
%Our supervision is in turn analogous to concrete entities in a database:
%	we are not given the latent parse, but only the entities that
%	are ``true'' in the model.

%(supervised)
Supervised approaches to logical parsing prominently include
	\newcite{key:1996zelle-semantics},
	\newcite{key:2005zettlemoyer-semantics},
	\newcite{key:2005kate-semantics}, 
	\newcite{key:2007zettlemoyer-semantics}, 
	\textit{inter alia}.
For example, \newcite{key:2007zettlemoyer-semantics} learn a mapping from
	textual queries to a logical form.
This logical form importantly contains all the predicates and entities
	used in their parse.
We loosen the supervision required in these systems by allowing the parse to
	be entirely latent;
	the annotation of the grounded time neither defines, nor gives any
	direct cues about the elements of the parse, since many parses evaluate
	to the same grounding.
To demonstrate, the grounding for \tp{a week ago} could be
	described by specifying a month and day, or as a \tp{week ago}, or as \tp{last
	Friday} -- substituting today's day of the week for \tp{Friday}.
Each of these correspond to a completely different parse.

%(distantly supervised)
Recent work by \newcite{key:2010clarke-semantics} and 
	\newcite{key:2011liang-semantics} similarly relax supervision 
	to require only annotated answers rather than full logical forms.
For example, \newcite{key:2011liang-semantics} constructs a latent parse
	similar in structure to a dependency grammar, but representing a logical
	form.
Our proposed lexical entries and grammar combination rules can be thought of as
	paralleling the lexical entries and predicates, and the implicit combination 
	rules respectively in this framework.
Rather than querying from a finite database, however, our system must compare
	temporal expression within an infinite timeline.
Furthermore, our system is run using neither lexical cues nor intelligent
	initialization.

% -- Rule-Based Temporal Comparison
%(rule based)
Related work on interpreting temporal expressions has focused on constructing
	hand-crafted interpretation rules
	% Random, TERESO, Marta's Friend, Edinburgh
	\cite{key:2000mani-temporal,key:2003saquete-temporal,key:2004puscasu-temporal,key:2010grover-temporal}.  
Of these, \sys{HeidelTime} \cite{key:2010strotgen-temporal} and
	\sys{SUTime} \cite{key:2012chang-temporal} provide particularly strong
	competition.

%(probabilistic)
Recent probabilistic approaches to temporal resulution include
	\newcite{key:2010uzzaman-temporal},    %TRIPS/TRIOS - probabilistic
	who employ a parser to produce deep logical forms, in conjunction with
	a CRF classifier.
In a similar vein,
	\newcite{key:2010kolomiyets-temporal} %KUL - partly probabilistic
	employ a maximum entropy classifier to detect the location and temporal
	type of expressions; the grounding is then done via deterministic rules.
	
